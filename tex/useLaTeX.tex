\documentclass[UTF8]{ctexbook}

\usepackage{graphicx}
\usepackage{bookmark}
\usepackage{listings}
\usepackage{circuitikz}
\lstset{numbers=left}

\title{\LaTeX{} 使用记录}
\author{ooohpi}
\date{\today}

\begin{document}
\maketitle
\tableofcontents
\chapter{熟悉 \LaTeX{}}
为啥要用\LaTeX{}??

\section{让\LaTeX{}跑起来}
\LaTeX{} 是有不同的发行版的,这里我选择的是 Tex Live 这个发行版。

可以通过一些镜像源来下载该软件,这里我选择的 \href{https://mirrors.tuna.tsinghua.edu.cn/help/CTAN/}{Tsinghua 镜像源}。

该软件的安装是挺简单的,设置一下安装路径,另外可以选择需要安装的宏包,这里我就没有选择,直接默认安装。

安装后,有一个 TLShell TeX Live Manager 软件,可以设置镜像源,并更新或者下载新的宏包。

安装了 TeX Live 后,由于我是用的 VSCode,所以需要安装一个扩展 LaTeX Workshop。这样基本就能够用了。

通过一小段代码来检验上面软件安装后的效果。

\begin{lstlisting}[language=Tex]
  \documentclass{article}

  \begin{document}
    Hello \LaTeX !
  \end{document}
  
\end{lstlisting}

由于 \LaTeX{} 是面向西文的,所以对于中文,需要使用一些宏包帮助我们设置或加载所需的字体、编码等。

\begin{lstlisting}[language=TeX, escapechar=']
  \documentclass[UTF8]{ctexart}

  \begin{document}
    '你好,'\LaTeX{}!
  \end{document}
\end{lstlisting}

\section{从一个例子说起}
\subsection{确定目标}
对于目标的话,你可能需要写一篇小论文、一本书、笔记等等。那么不同的目标,就需要我们确定要使用什么语言及哪些文档类、宏包等,也可以大致确定文档内容的构成元素,如正文、图标、代码、引用等等。

\subsection{从提纲开始}
在确定了目标过后,就可以对 \LaTeX{} 进行大致的设置,来满足我们的目标需求。

\begin{lstlisting}[language=Tex]
  %-*- coding: UTF8 -*-
  % gougu.tex
  
  \documentclass[UTF8]{ctexart}

  \title{xxx}
  \author{xx}
  \date{\today}

  \bibliographystyle{plain}

  \begin{document}
    \maketitle
    \tableofcontents
    \section{xxx}
    \section{ppp}
    \bibliography{math}

  \end{document}
\end{lstlisting}

这里有几处使用了注释,在 \LaTeX{} 中注释通过在行首使用 \% 实现,注释可以帮助我们以后更好的理解文档的内容和设置。相信学习过编程的同学应该有听过注释的重要性。

文档类,通过 documentclass 命令来设置。

标题、作者、日期等信息的声明。

document 环境,通过 begin 和 end 命令来声明。

document 环境前,通常称为导言区(preamble),bibliographystyle 命令设置了参考文献的格式,很多文档的设置是在导言区的。

通过 maketitle 命令实际输出标题。

通过 tableofcontents 命令输出目录。

section 命令开始新的一节。

bibliography 命令来从文献数据库里提取信息。

除了正文外,空格并无太大影响,可以通过其来使文档格式更受看。

比如目录这些内容,需要多次编译才能正常显示出来。

\subsection{填写正文}
空行分段,空格和单个换行并无太大作用。

文档代码里,段前不同缩进。

中文汉字和其他字符间使用空格分隔,可能更好。

\subsection{命令和环境}
脚注通过 footnote 命令实现。

emph 命令可以改变字体形状,表示强调。

命令都以 \textbackslash 开头,后接命令,命令可以带有参数,如果参数不止一个,需用花括号包裹起来,如果带有可选参数,则使用方括号包裹起来。注意命令与其他内容之间的分隔。
% 这里可以直接将命令格式通过表格什么的排版出来,更直观

引用通过声明 quote 环境来实现。

一些环境的预设可能并不能满足我们的要求,这是我们可以自己对其进行修改。

zihao 命令可以选择字号,kaishu 命令可以将字体切换为楷书。这些命令又叫声明,会对其后直到该分组结束的内容产生影响。

一个环境自然就是一个分组,可以通过 \{\} 产生一个分组。

环境声明的一般形式。环境可带参数和可选参数。

摘要通过 abstract 命令实现。

定理环境是一类环境??使用前需要在导言区进行定义??

\begin{lstlisting}[language=Tex]
  \newtheorem{thm}{mytherom} % code in preamble

  \begin{thm}[theromname]
    xxxx
  \end{thm}
  
\end{lstlisting}

\subsection{遭遇数学公式}
通过 \$ 将内容包裹起来,就可以在正文中插入数学公式。将这样写在正文中的公式称为正文公司或行内公式。

还有一种显示公式或者叫列表公式的方法在文档中显示公式。通过 equation 环境可以实现。

对于数学公式,一些符号并不能直接键入,这时可以通过命令来显示正确的符号。

数学公式不止是符号的堆砌,还有其结构,如上下标、分式、根式、矩阵等。

一些符号可能没有直接的命令,这时可以通过命令和结构的组合来达到想要的效果。如 $^\circ$。

对于公式的键入,通过将常用命令熟记,应可以较好的完成公式代码,不过也可以考虑一些自动化工具,来通过识别以输出 \LaTeX 数学公式代码的目的。

\subsection{使用图表}
对于图片,可以使用已经准备好的图,或者在文档源码里通过代码绘制,后者一般较难。

插图功能需要导入宏包来实现,例如 graphicx 。通过 usepackage 命令可以导入宏包,导入宏包后,即可利用宏包提供的命令实现目标。

graphics 提供了 includegraphics 命令来插入图片。通过对该命令的可选参数的设置,可以调整加入图片的效果。

一般没有直接插入图片,而是在单独的环境里插入图片,如 figure 环境,可通过可选参数对其进行设置,也可在环境里使用其他命令,如给图片添加标题、序号、标签等,以免和正文产生不想要的效果,当然直接插入满足目的也可,得灵活安排,是吧。

对于表格,一般直接通过 \LaTeX 代码实现,如 tabular 环境,而且其通常放置于 table 环境中。
\begin{lstlisting}[language=Tex]
  % need use package 'float' in preamble
  \begin{table}[H] 
    \begin{tabular}{|rrr|}
      \hline
      $a$ & $b$ & $c$ \\
      \hline
      3 & 4 & 5 \\
      5 & 12 & 13 \\
      \hline
    \end{tabular}%
    \qquad
    $a^2 + b^2 = c^2$
  \end{table}
\end{lstlisting}

通过 qquad 命令可以产生 2em 的空格。

上述 figure 和 table 环境有叫做浮动环境。上面 table 环境的特殊参数 H 是由 float 宏包实现的,使得 table 环境不浮动。

\subsection{自动化工具}
自动化工具处理的一大任务就是参考文献列表。

通过 BibTeX 处理参考文献列表,还需要准备其参考文献数据库,后缀为 bib 的文件,该文件一般不是人工输入,而是通过别的文献管理工具生成。

通过 bibliographystyle 命令声明了参考文献列表的格式,bibliography 命令读取参考文献数据库,并打印参考文献列表。但还需要在文章中通过 cite 命令来引用文献,该命令会在引用位置处显示该引用文献在参考文献列表中的编号。如果不要直接引用,可以通过 nocite 命令将文献显示在参考文献列表,一般将这些命令放置与 bibliography 命令之前。

其他的较简单的自动化工具还有页码、定理公式等的自动编号、图标公式等的交叉引用。

目录通过 tableofcontents 命令显示。

通过 label 命令将对象设置标签,就可通过基本的交叉引用命令 ref 对其进行引用。其他一些引用命令可能显示更好的效果,不过需要使用别的宏包,如 amsmath 提供了 eqref 命令,提供更好的公式引用效果。

\subsection{设计文章格式}
在完成上述过程后,会发现编译后的文档效果并不是我们想要的,如字体、字号、页面大小、页边距等等,不过好在有一些宏包可以帮助我们完成这些格式的设计。一些简单的设置可以放置在导言区,如果比较复杂了,我觉得还是通过模板进行设计可能更好,也便于以后的持续改进和利用。

设计页面尺寸可使用 geometry 宏包。

设计图表标题格式可使用 caption 宏包。

增加目录项目可使用 tocbibind 宏包。

可以通过 newenvironment 命令定义新的环境,在后续代码中使用新的环境,来是的其结构更加简洁。
\begin{lstlisting}[language=Tex]
  \newenvironment{myquote}
    {\begin{quote}\kaishu\zihao{-5}}
    {\end{quote}}
\end{lstlisting}

可以通过 newcommand 命令定义新的命令。
\begin{lstlisting}[language=Tex]
  \newcommand{\degree}{^\circ}
\end{lstlisting}


\chapter{组织你的文本}

\chapter{自动化工具}

\chapter{玩转数学公式}

\chapter{绘制图表}

\chapter{幻灯片演示}

\chapter{从错误中救赎}

\chapter{\LaTeX{}无极限}

\end{document}